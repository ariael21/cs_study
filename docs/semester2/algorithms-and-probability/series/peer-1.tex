\documentclass[a4paper]{article}
\usepackage[a4paper, total={7in, 10in}]{geometry}

%\usepackage[T1]{fontenc}
%\usepackage[utf8]{inputenc}
\usepackage{tgadventor}
\usepackage{amsmath,amssymb}
\usepackage{graphicx}

\renewcommand{\familydefault}{\sfdefault}

\begin{document}

    \graphicspath{ {./images/} }
    \section*{Algorithmen und Wahrscheinlichkeit: Peer 1}
    Yael Fassbind
    \section*{Aufgabe 1 - Touren}
    \subsection*{a) Finden Sie einen minimalen Spannbaum T von G. Begründen Sie Ihre Schritte/Rechnungen genau}
    Um den MST zu finden wurde Prim's Algorithmus verwendet:
    \begin{itemize}
        \item Start bei einem beliebigen Knoten in $G$. Diesen Knoten markiert man als besucht und fügt ihn zum Zusammenhangskomponenten (ZSHK) hinzu.
        \item Man schaut alle ausgehenden Kanten des ZSHK an, welche noch nicht im MST sind. Von diesen wählt man die Kante mit den geringsten Kosten.
        \item Die ausgewählte Kante fügt man dem MST hinzu. Der Knoten (welcher über die gewählte Kante erreicht werden kann) wird dem ZSHK hinzugefügt.
        \item Wiederhole diese Schritte, bis alle Knoten von $G$ im ZSHK sind.
    \end{itemize}
    \includegraphics[width=\linewidth]{p1-1}

    \pagebreak
    \subsection*{b) Bezeichne C das Gewicht von T. Verwenden Sie T um einen Zyklus Z zu finden, der in A beginnt und endet, jeden Knoten von G mindestens einmal besucht und Gewicht höchstens 2C hat}
    Der MST $T$ von $G$ hat ein Gewicht von C = 30 [Siehe a)]
    \\\\
    \textbf{Zyklus Z in G finden:}
    \begin{itemize}
        \item Finde MST T in G (siehe a))
        \item Verdopple alle Kanten von T $\Rightarrow$ T'\\
        \linebreak
        \includegraphics[width=200pt]{p1-2}
        \item Finden die Eulertour E in T'\\
        \linebreak
        \includegraphics[width=250pt]{p1-3}
        \item Kürze E, indem man Konten die mehrfach besucht werden überspringt. Dazu wird die Dreiecksungleichung verwendet.\\
        \\ 
        Dreiecksungleichung: $c(x,z) \le c(x,y) + c(y,z)$\\
        \linebreak
        \includegraphics[width=400pt]{p1-4}
    \end{itemize}
    Daraus folgt das Gewicht von $Z$ folgendermassen:\\
    $Z \le 3 + 4 + 2 + 10 + 5 + 20 + 16$\\
    $Z \le 60 = 2 * 30 = 2 * C$\\
    $\Rightarrow Z \le 2 * C$
    
    \pagebreak
    \subsection*{c) Beweisen Sie, dass jeder Zyklus Z, der jeden Knoten mindestens einmal besucht, mindestens Gewicht C hat.}
    Wir haben Graph $G = (V, E)$, dessen MST mit Gewicht $C$ und ein Zyklus $Z$ der alle Knoten von $G$ mindestens einmal besucht.\\
    \\
    Das Gewicht des MST von $G$ ist genau $C$. Daher sind die minimalen Kosten um alle Knoten zu erreichen genau $C$. Desswegen wird jeder andere Weg der alle Knoten besucht \textbf{mindestens $C$ Kosten.}\\
    \\
    Wenn man von $Z$ eine beliebige Kante $e \in E$ entfernt, wird Z zu einem Pfad welcher alle Knoten von $G$ mindestens einmal besucht. Dieser hat mindestens Kosten $C$ (entsprechend der vorherigen Argumentation), folgendermassen \textbf{hat $Z$ auch mindestens Kosten $C$.}

\end{document}

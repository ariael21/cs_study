\documentclass[a4paper]{article}
\usepackage[a4paper, total={7in, 10in}]{geometry}

\usepackage[T1]{fontenc}
\usepackage[utf8]{inputenc}
\usepackage{tgadventor}
\usepackage{amsmath,amssymb}
\usepackage{graphicx}

\renewcommand{\familydefault}{\sfdefault}



\begin{document}

    \graphicspath{ {./images/} }
    \section*{Algorithmen und Wahrscheinlichkeit: Serie 2}
    Kenli Lao und Yael Fassbind
    \subsection*{Aufgabe 1 - Längste Pfade}
    a) Konstruiere einen Graph G' = (V', E') aus G=(V, E), sodass gilt:
    \\\\
    Es gibt einen Hamiltonpfad in G welcher $e \in E$ benutzt $\Longleftrightarrow $ der längste Pfad in G' hat die Länge $|V'| -1$
    \\\\
    Konstruktion von G' = (V', E'):
    \\
    \includegraphics[width=\linewidth]{s2-1}
    \\
    Graph G' hat einen längsten Pfad von länge $|V'| - 1$, da er alle Knoten besucht. 
    \\\\
    Rightung: $\Rightarrow$
    \\
    $\Rightarrow$ Da G einen Hamiltonkreis hat der durch $e \in E$ geht, gibt es einen Kreis in G der alle Knoten besucht und durch $e$ geht. Wenn $e = \{u,v\}$, dann gibt es einen längsten Pfad von u nach v der länge $|V| - 1$, weil es einen Hamiltonkreis gibt der durch u und v geht.
    \\
    $\Rightarrow$ Wenn nun G' wie oben beschrieben konstruiert wird, hat G' einen neuen längsten Pfad: der Pfad von x nach y. Dieser hat genau die Länge $|V'| - 1$, da er durch den längsten Pfad von G geht aber zusätzlich bei den zwei neuen Knoten startet.

    [...]
\end{document}

\documentclass[a4paper]{article}
\usepackage[a4paper, total={7in, 10in}]{geometry}

\usepackage[T1]{fontenc}
\usepackage[utf8]{inputenc}
\usepackage{tgadventor}
\usepackage{amsmath,amssymb}


\usepackage{xcolor}
\usepackage{tcolorbox}

\usepackage{multicol}
\setlength{\columnsep}{1cm}

\renewcommand{\familydefault}{\sfdefault}

\tcbset{width=1.01\textwidth, boxrule=1pt, colframe=black!75, arc=0pt, arc=0mm,left=2pt, right=2pt, boxsep=2pt}


\title{252-0030-00L Algorithmen und Wahrscheinlichkeit\\Summary and Lecture Notes}
\author{Yael Fassbind}
\date{FS 2021}

\begin{document}
\maketitle

\begin{center}
    Disclaimer: These are just my notes during the semester.\\
    No guarantee for completeness and correctness
\end{center}



\pagebreak
\section{Week 1}
\subsection{Zusammenhang}

\begin{tcolorbox}[title=Definition]
    Sei $G = (V,E)$ ein Graph. 

    G heisst zusammenhängend, wenn $\forall u,v \in V, u \neq v$ gilt: es gibt einen u-v-Pfad in G.
\end{tcolorbox}

\begin{tcolorbox}[title=Definition]
    Sei $G = (V,E)$ ein Graph. $u,v \in V, und  X \subseteq V \backslash {u,v}$

    X heisst , wenn u und v in verschiedenen Zusammen-hangskomponenten von $G[V \backslash X]$ liegen. G heisst k-zusammenhängend, wenn gilt:

    \begin{itemize}
        \item $|V| \geq k + 1$
        \item $\forall u,v,\in V:$ jeder u-v-Seperator X hat Grösse $|X| \geq k$ 
    \end{itemize}
\end{tcolorbox}

Einfache Erklärung: Um einen k-zusammenhängenden Graph zu zerstören, muss man k Knoten löschen.
\\

\begin{tcolorbox}[title=Definition: Satz von Menger (Knoten)]
    Sei $G = (V,E)$ ein Graph. $u,v \in V$ dann gilt:

    Jeder u-v-Separator X hat Grösse $|X| \geq k \Longleftrightarrow$ es gibt k intern-knotendisjunkte u-v-Pfade. 

\end{tcolorbox}


\subsection*{Glossary}
bipartie: 


\end{document}